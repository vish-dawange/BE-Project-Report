\chapter{Research Methodology}


\section{Steps to acquire and process}
\subsection{Parser & Tokenize (Flex)}
This stage helps us token the program for further processing.

\subsection{Grammar & Generation of tables (Yacc)}
Once the tokens are generated we pass the output to this phase, where it goes
through our grammar, which helps find out different threads, functions associated
with them, parameters, shared memory being used in them.
We generate the following tables based on the grammar:-
\begin{itemize}
\item Global variable (var\_name, location)
\item User-defined Functions (func\_name, location)
\item Parameter table (func\_name, all\_parameters)
\item Thread-Functions table (thread\_handle, func\_name)
\item Function local variable table (func\_name, var\_name)
\item Variables' log table (var\_name, func\_name, location)
\item Critical Section table (var\_name, location)
\end{itemize}
\subsection{Analysis}
Once the tables are generated, we check if any global variables are being used
in multiple functions, and whether they are handled using sem\_wait() and sem\_post()
function calls.

