\begin{thebibliography}{99}
\addcontentsline{toc}{chapter}{\bibname}
\lhead{}\markboth{\bibname}{}
% \bibitem{TEXT} is how you refer to your reference in your report. keep that very short
% \emph{Paper Name} is just to highligh the paper name by making it italic, not required but looks nice
   
%Example
%\bibitem{short_paper_name}\emph{Paper Name}; Author Name, Conference Name, year etc

\bibitem\emph{"Automatic Critical Section Discovery using Memory Usage Patterns"}; Lisa Marie Stechschulte, Department of Electrical and Computer Engineering, University of Maryland, 2012
\bibitem\emph{"Race Condition Detection for Debugging Shared Memory in Parallel Programs"}; Robert Harry Benson Netzer, University of Wisconsin - Madison, 1991
\bibitem\emph{"Automatic Lock Insertion in Concurrent Programs"}; Tolubaeva, M., Computational Intelligence for Modelling Control and Automation, 2008 International Conference, 10.1109—CIMCA.2008.225
\bibitem\emph{"Minimum Lock Assignment: A Method for Exploiting Concurrency among Critical Sections"}; Yuan Zhang, Vugranam C. Sreedhar, Weirong Zhu, Vivek Sarkar, and Guang R. Gao, Languages and Compilers for Parallel Computing, DOI - 10.1007—978-3-540-89740-8-10, Print ISBN - 978-3-540-89739-2
\bibitem\emph{"Multithreaded Programming Framework Development for gcc infrastructure"}; Dr. Niranjan N. Chiplunkar, B. Neelima, Mr. Deepak. 978-1-61284-840-2/11/ ©2011 IEEE
\bibitem\emph https://computing.llnl.gov/tutorials/pthreads/

\bibitem\emph {GNU Compiler Collection}; http://gcc.gnu.org
\bibitem\emph {GNU Compiler Collection Internals} http://gcc.gnu.org/onlinedocs/gccint/
\bibitem\emph http://en.wikipedia.org/
     
\end{thebibliography}
